Unison is a file-synchronization tool for POSIX-compliant systems (e.g. BSDs,
GNU/Linux, macOS) and Windows.  It allows two replicas of a collection of files
and directories to be stored on different hosts (or different disks on the same
host), modified separately, and then brought up to date by propagating the
changes in each replica to the other.

Features:
\begin{itemize}
\item Unison works {\em across} platforms, allowing you to synchronize a
  Windows laptop with a Unix server, for example.
\item Unlike a distributed filesystem, Unison is a user-level program:
  there is no need to modify the kernel or to have
  superuser privileges on either host.
\item Unlike simple mirroring or backup utilities, Unison can deal
  with updates to both replicas of a distributed directory structure.
  Updates that do not conflict can be propagated automatically.
  Conflicting updates are detected and displayed.
\item Unison works between any pair of machines connected to the
  internet, communicating over either a direct socket link or
  tunneling over an encrypted {\tt ssh} connection.
  It is careful with network bandwidth, and runs well over slow links.
  Transfers of small updates to large files are optimized using a compression
  protocol similar to rsync.
\item Unison has a clear and precise specification\iffull, described
below. \else. \fi
  \item Unison is resilient to failure.  It is careful to leave the
  replicas and its own private structures in a sensible state at all
  times, even in case of abnormal termination or communication
  failures.
% \item Unison is easy to install.  Just one executable file (for each
%   host architecture) is all you need.
\item Unison is free; full source code is available under the GNU
Public License.
\end{itemize}
